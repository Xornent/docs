% !TEX TS-program = xelatex
% !TEX encoding = UTF-8 Unicode

% This is a simple template for a LaTeX document using the "article" class.
% See "book", "report", "letter" for other types of document.

\documentclass[11pt]{article}
% use larger type; default would be 10pt

\usepackage[utf8]{inputenc} 
% set input encoding (not needed with XeLaTeX)

%%% Examples of Article customizations
% These packages are optional, depending whether you want the features they provide.
% See the LaTeX Companion or other references for full information.

%%% PAGE DIMENSIONS
\usepackage{geometry} 
% to change the page dimensions
\geometry{a4paper} 
% or letterpaper (US) or a5paper or....
% \geometry{margin=2in} % for example, change the margins to 2 inches all round
% \geometry{landscape} % set up the page for landscape
% read geometry.pdf for detailed page layout information

\usepackage{chemfig}
\usepackage{graphicx} 
% support the \includegraphics command and options
% \usepackage[parfill]{parskip} 
% Activate to begin paragraphs with an empty line rather than an indent

%%% PACKAGES
\usepackage{booktabs}   % for much better looking tables
\usepackage{array}      % for better arrays (eg matrices) in maths
\usepackage{paralist}   % very flexible & customisable lists (eg. enumerate/itemize, etc.)
\usepackage{verbatim}   % adds environment for commenting out blocks of text & for better verbatim
\usepackage{subfig}     % make it possible to include more than one captioned figure/table in a single float
% These packages are all incorporated in the memoir class to one degree or another...

%%% HEADERS & FOOTERS
\usepackage{fancyhdr} % This should be set AFTER setting up the page geometry
\pagestyle{fancy}     % options: empty , plain , fancy
\renewcommand{\headrulewidth}{0pt}
\lhead{}\chead{}\rhead{}
\lfoot{}\cfoot{\thepage}\rfoot{}

%%% SECTION TITLE APPEARANCE
\usepackage{sectsty}
% \allsectionsfont{\sffamily\mdseries\upshape} 
% (See the fntguide.pdf for font help)
% (This matches ConTeXt defaults)

%%% ToC (table of contents) APPEARANCE
% \usepackage[notlof,notlot]{tocbibind} 
% Put the bibliography in the ToC
\usepackage[titles,subfigure]{tocloft}
% Alter the style of the Table of Contents
\renewcommand{\cftsecfont}{\rmfamily\mdseries\upshape}
\renewcommand{\cftsecpagefont}{\rmfamily\mdseries\upshape} 

%%% END Article customizations

%%% DOCUMENT CONTENT
\title{\textbf{ 
  Nomenclature of Organic Compounds
  }}
\author{\emph{@yang-z} Revision 2}
% \date{} % Activate to display a given date or no date (if empty),
          % otherwise the current date is printed 

\begin{document}
\maketitle
\setlength{\parskip}{0.8 \baselineskip}

\section{Contribution}

This document is licensed under \textbf{GPLv3}. And is compiled with \textbf{TexLive}.
The source code is avaliable on https://github.com/xornent/docs/ location \textbf{chem.org/nomenclature}.
Contribution to the document and error report by \textbf{Issues} are welcomed.

Install TexLive parameters
\ttfamily
apt-get install: texlive-latex-base latex-cjk-all texlive-latex-extra
\rmfamily

\section{Number Prefix}

\begin{table}[htbp]
	\centering
	\caption{Universal Number Prefixes}
	\begin{tabular}{cc|cc|cc}
		\toprule 
		Index & Expression & Index & Expression & Index & Expression \\
		\midrule 
		 1 &   (mono) &  2 &     di &  3 &      tri \\
		 4 &   tetra &  5 &   penta &  6 &    hexa \\
		 7 &   hepta &  8 &    octa &  9 &    nona \\
	    10 &    deca & 11 &  undeca & 12 &  dodeca \\   
    \toprule
    \end{tabular}
\end{table}

For carbon atoms in a chain less than 10:
\begin{table}[htbp]
	\centering
	\caption{Table of Carbon Number Prefixes}
	\begin{tabular}{cc|cc|cc}
		\toprule 
		Index & Expression & Index & Expression & Index & Expression \\
		\midrule 
		 1 &    meth &  2 &     eth &  3 &    prop \\
		 4 &    buta &  5 &   penta &  6 &    hexa \\
		 7 &   hepta &  8 &    octa &  9 &    nona \\
	    10 &    deca & 11 &  undeca & 12 &  dodeca \\   
    \toprule
    \end{tabular}
\end{table}

When there are more than 10 carbon atoms in a chain, the ones place of the number is 
expressed first, then the tens place. There are minor changes to the expression for
digits at ones place (Table 2), and for those on tens place (Table 3). However, it is
exceptional for number 11 (as undeca).

\begin{table}[htbp]
	\centering
	\caption{Digits at Ones Place in Numbers Greater Than 10}
	\begin{tabular}{cc|cc|cc}
		\toprule 
		Index & Expression & Index & Expression & Index & Expression \\
		\midrule 
		 1 &     hen &  2 &      do &  3 &     tri \\
		 4 &   tetra &  5 &   penta &  6 &    hexa \\
		 7 &   hepta &  8 &    octa &  9 &    nona \\
    \toprule
    \end{tabular}
\end{table}

\begin{table}[htbp]
	\centering
	\caption{Digits at Tens Place}
	\begin{tabular}{cccrrr}
		\toprule 
		Index & Expressing Itself & Prefix & +1 & +2 & +3    \\
		\midrule 
		 1 &      deca &   deca &    undeca &  dodeca & trideca \\
		 2 &    eicosa &   cosa & heneicosa &  docosa & tricosa \\
		 3 & triaconta & triaconta & hentriaconta & dotriaconta & tritriaconta \\
		 4 & tetraconta & tetraconta & hentetraconta & dotetraconta & tritetraconta \\
		... & ... & ... & ... & ... & ... \\
    \toprule
    \end{tabular}
\end{table}

\section{Relative Positioning}

Some prefixes are used to identify the relative position of specified functional groups,
spatial isomers, and common names for organic compounds 

\begin{table}[htbp]
	\centering
	\caption{List of Relative Positioning}
	\begin{tabular}{ll}
		\toprule 
		Prefix & Descriptions \\
		\midrule 
        iso			&	Isotetrane: \(CH_{3}CHCH_{3}CH_{3}\) \\
		neo			&	Neopentane: \(C(CH_3)_4\) \\

		\textbf{Relative Positions} & \\
		ortho (o-)	&	Indicating the two functional groups are adjacent in a ring \\
		meta (m-)	&	The two functional groups was separated \\
		para (p-)	&	The two functional groups are at their opposite, especially\\ & in an aromatic compound \\
		primary		&	An atom connecting to just one carbon atom \\
		secondary (sec-) & An atom connecting to two carbon atoms \\
		tertiary (tert-) & An atom connecting to three carbon atoms \\
		quaternary	&	An atom connecting four carbon atoms \\

		\textbf{Geometric Isomerism} & \\
		cis 		&	\\
		trans		&	\\
    \toprule
    \end{tabular}
\end{table}

\section{Hydrocarbons}
\subsection{Acyclic Hydrocarbons}

Get the number of carbon atoms in the hydrocarbons, and check for its number prefix.
Alkane names are the number prefixes (or with their modifers) plus -ane. Alkene name
suffix is -ene, and alkyne name suffix is -yne. You should omit the last letter 
\textbf{a} if the prefix ends with one.

\begin{table}[htbp]
	\centering
	\caption{Naming Main Chains of Hydrocarbons}
	\begin{tabular}{cclll}
		\toprule 
		Carbon Numbers & Prefix & Alkanes & Alkenes & Alkynes \\
		\midrule 
        1   & Meth- & Methane & - & - \\
		2	& Eth-  & Ethane & Ethene & Ethyne (Common Name: Acetylene) \\
		3	& Prop- & Propane & Propene & Propyne \\
		4	& Buta- & Butane & Butene & Butyne \\
    \toprule
    \end{tabular}
\end{table}

For compounds with more than one double bonds or triple bonds, the corresponding
alkenes and alkynes have suffixes like \textbf{Universal Number Prefix (Mono omitted)
 - Functional Group Suffix }. 

\begin{table}[htbp]
	\centering
	\caption{Hydrocarbons with Multiple Functional Groups}
	\begin{tabular}{cll}
		\toprule 
		Functional Groups & Alkenes & Alkynes \\
		\midrule 
        1   & -ene & -yne \\
		2   & -diene & -diyne \\
		3	& -triene & -triyne \\
		... & ... & ... \\
    \toprule
    \end{tabular}
\end{table}

\setchemfig {
	bond sep = 1.1pt,
	atom sep = 1.1em,
	bond style = {line width = 0.5pt}
}

\subsubsection{Chirality Isomerism}

Given that a carbon atom is connected to four groups of atom \(R_1, R_2, R_3, R_4\). 
The chilarity of that carbon is determined using the following technique.

\begin{itemize}
	\item If none of the groups are identical, the carbon is chiral carbon atoms, otherwise it is not.
	\item Look at the carbon atom in a position that the smallest of the four groups is hidden from you, and you are faced with three groups.
	\item Make an arrangement according to the size of the three groups, count them from the largest to the smallest. If the sequence is \textbf{clockwise}, the carbon is a \textbf{R}(rectus) carbon, if \textbf{counterclockwise}, \textbf{S}(sinister) carbon.
\end{itemize}

It is essential to know the relative size of each functional group, it is assessed and
compared using the \textbf{Cahn-Ingold-Prelog Sequence Rule}. This rule is rather
complex in order to achieve completeness and correctness, you can see Common Functional
Group Section for a brief introduction of this rule.

\subsubsection{Geometry Isomerism of Ethenes}

Arrange the two groups connecting to the either side of the ethene C=C, if the two larger
ones are at the same side of linear core, they are called \textbf{Z}(zusamman) isomers,
otherwise, they are called \textbf{E}(entgegen) isomers.

They are marked in the beginning of the name, using Z-n-nomenclature, E-n-nomenclature, or
(nZ,mZ, ... ,kE,pE, ...)-nomenclature. For example, butene has two isomers \textbf{Z-2-butane}, \textbf{E-2-butane}

\subsection{Alicyclic Hydrocarbons}

Alicyclic hydrocarbons are named with prefix cyclo- plus the name of the corresponding
acyclic hydrocarbons. For example, \(C_{6}H_{14}\) \textbf{Hexane} becomes
\(C_{6}H_{12}\) \textbf{Cyclohexane}.

\subsubsection{Geometry Isomerism of Alicyclic Hydrocarbons}

If the molecule has symmetrical properties over an axis, simply name the geometry isomer
by \textbf{cis} and \textbf{trans} prefix, otherwise, specify all the chiral carbon atoms
using its index and \textbf{R/S} types.

For example, \chemfig{*3((-H_3C)(-H)-(-CH_3)(-[:60]H)<>)} \textbf{cis-1,2-dimethylcyclopropane} or 
\textbf{(1S,2R)-1,2- dimethylcyclopropane}.

\chemfig{*3((-H_3C)(-H)-(-[:60]CH_3)(-H)<>)} \textbf{(1R,2R)-1,2-dimethylcyclopropane}.

\chemfig{*3((-H)(-H_3C)-(-[:60]H)(-CH_3)<>)} \textbf{(1S,2S)-1,2-dimethylcyclopropane}.

\subsection{Hydrocarbons as Functional Groups}

Replacing the -ne ending of hydrocarbons to -yl to express hydrocarbons as functional
groups, and to express hydrocarbons with branched chains. For example, \(CH_{4}\) 
\textbf{Methane} becomes \(CH_{3}-\) \textbf{Methyl-}, \(CH_2=CH_2\) \textbf{Ethene}
becomes \(CH_2=CH-\) \textbf{Ethenyl}, and \textbf{-yne} becomes \textbf{-ynyl}.

\subsection{Hydrocarbons with Branched Chains}

One can name the hydrocarbons with the following rules to name and arrange the branched
chains. Note that the rules must be applied in specified order.

\begin{itemize}
	\item \textbf{Step I} Find the longest chain of the hydrocarbon.
	\item \textbf{Step II} If there are more than one chain that has the maximal length, choose the one with most branch chains.
	\item \textbf{Step III} Mark the index from the either end of the chain, and picks out the indices that the branch chains link. These mark sequence should \dots
	\item \textbf{Step IV} Select the marking method that makes the first index smallest.
	\item \textbf{Rule V} If the first index is identical, select that makes the second index smallest.
	\item \textbf{Rule VI} If all are identical, arrange the names of each substituent, and make the first one in alphabetical order the smallest.
\end{itemize}

Make the longest branch as the main chain, and to choose the orientation to mark The
index of carbons in the main chain, which makes the sum of all branched chain index
to be smallest. \textbf{(The same as Chinese nomenclature)}

Note that \textbf{the arrangement of the functional groups and branched chains are
according to the ascending alphabetical order, but not the size of the chain which
applied in Chinese nomenclature}. (When evaluating the sequence, we should not take
number prefixes into consideration, so \textbf{-Ethyl} is previous than \textbf{-Dimethyl})

\begin{table}[htbp]
	\centering
	\caption{List of Sample Branched Hydrocarbons}
	\begin{tabular}{ll}
		\toprule 
		Chemical Structure & Nomenclature \\
		\midrule 
        \((CH_3)_2CH_2CH_2CH(C_3H_7)CH(C_2H_5)CH(CH_3)_2\) &
		    3-ethyl-2,7-\textbf{di}methyl-4-\textbf{propyl}oct\textbf{ane} \\
		\(C(CH_3)_4\) & 
			2,2-\textbf{di}methyl\textbf{prop}ane (or neopentane) \\
		
		\textbf{Nested Structures} & \\
		\(CH_3(CH_2)_2CH(CH_3)CH(CH(CH_3)_2)(CH_2)_2CH_3\) &
			4-\textbf{iso}propyl-5-\textbf{methyl}oct\textbf{ane} \\
		& 4-\textbf{methyl}-5-(1-\textbf{methyl}ethyl)\textbf{oct}ane \\
    \toprule
    \end{tabular}
\end{table}

Branched alkenes and alkynes are marked in a way that the double bond and triple
bond index is smallest. For example \(CH_2=C(CH_3)CH=CH_2\) is named as \textbf{
	2-methyl-1,3-butadiene
}

When the compounds have both double bonds and triple bonds, its name should contain 
both alkene endings -en and alkyne ending -yne. if there is no ambiguation, one can
use -enyne instead of the two endings with their positions specified explicitly.

\begin{table}[htbp]
	\centering
	\caption{List of Sample Branched Hydrocarbons}
	\begin{tabular}{ll}
		\toprule 
		Chemical Structure & Nomenclature \\
		\midrule 
        \(HC\equiv C-CH=CH_2\) & butenyne \\
		\(CH_3CH=CHC\equiv CH\) & 3-penten-1-yne \\
    \toprule
    \end{tabular}
\end{table}

If the compounds have alicyclic hydrocarbonyl components, other branches should be 
seen as branched chains, naming before the cyclic hydrocarbons. For example, there
is a legal name: \textbf{2-methyl-1-(3-methylbutyl)cyclohexane}, which is a derivative
of cyclohexane.

\section{Aromatic Hydrocarbons}

The basis of aromatic hydrocarbons are \(C_6H_6\) \textbf{Benzene} and other condensed
aromatic hydrocarbons. \(C_{10}H_8\) \textbf{Naphthalene} \(C_{14}H_{10}\) \textbf{Anthracene}
These cores are usually placed at the end of the word.

The differences in geometry formed geometric isomerism in aromatic hydrocarbons,
for example, \(Ph-(CH_3)_3\) has three isomers, \textbf{1,3,5-trimethylbenzene} or 
para-trimethylbenzene, \textbf{1,2,4-trimethylbenzene} or meta-trimethylbenzene, and
\textbf{1,2,3-trimethylbenzene} or ortho-trimethylbenzene. (Table 5).

\section{Halogenides}

\begin{table}[htbp]
	\centering
	\caption{Halogens and Halides}
	\begin{tabular}{cll}
		\toprule 
		Chemical & Halo & Halide \\
		\midrule 
        \(F\)  & Fluoro & Fluoride \\
		\(Cl\) & Chloro & Chloride \\
		\(Br\) & Bromo & Bromide \\
		\(I\) & Iodo & Iodide \\
    \toprule
    \end{tabular}
\end{table}

\begin{table}[htbp]
	\centering
	\caption{List of Common Names}
	\begin{tabular}{cl}
		\toprule 
		Chemical & Nomenclature \\
		\midrule 
        \(CHCl_3\)  & Chloroform \\
		\(CHBr_3\) & Bromoform \\
		\(CCl_2F_2\) & Freon \\
    \toprule
    \end{tabular}
\end{table}

The table \textbf{Halogens and Halides} and \textbf{List of Common Names} of halogenides listed some
basic naming principles, in another circumstance, the halogen atom replace the groups directly
linked to a carbonyl group. \chemfig{C(=[:90]O)(-[:-30])(-[:-150])}. They are regarded as
a derivative of carboxylic acids, and are named by replacing the \textbf{-ic acid} ending to
\textbf{-yl halide}. For example \textbf{Propanoic acid} became \textbf{Propanoyl chloride}.

\section{Alcohols and Phenols}

A hydroxyl functional group attached to a chain is called alcohols, and to name it, Put
an -ol ending to the original name of \textbf{hydrocarbons}, and omit the last letter e if there is any. For
compounds that have more than one hydroxyl groups, the ending becomes \textbf{-diol},
\textbf{-triol} and so forth. (Do not omit the e when there are multiple hydroxl groups.)

Phenols are those with hydroxl groups attached directly to an aromatic hydrocarbonyl ring.
and exceptionally, the simpliest phenol \(Ph-OH\) is nicknamed \textbf{Phenol}, while
other derivatives are named using the rules above. For example, \(HO-C_6H_4-OH\) is named
\textbf{Benzenediol}

\textbf{When naming the alcohols, we selects the main chain that:}
\begin{itemize}
	\item The hydroxl function group is \textbf{directly linked} to one of the chain's carbons.
	\item The chain which match the rule above is longest.
	\item Mark the index of carbons from the hydroxl group.
\end{itemize}

\begin{table}[htbp]
	\centering
	\caption{Alcohols and Phenols}
	\begin{tabular}{cl}
		\toprule 
		Chemical & Nomenclature \\
		\midrule 
        \(C(CH_3)_3OH\)  & 2-methyl-2-propanol (derived from \textbf{propane}) \\
		\(CH_3CH=C(C_2H_5)CH_2OH\) & 2-ethyl-2-butenal \\
		\(CH_2OH-CHOH-CH_2OH\) & 1,2,3-propanetriol (\textbf{glycerin}) \\
    \toprule
    \end{tabular}
\end{table}

Notice that if there are more than 3 hydroxyl functional groups, (a chain can have
at most 2 hydroxyls at both ends) one group must not be in the main chain. For them
we name it as \textbf{hydroxo-}. For example \(HOCH_2CH_2CH(CH_2OH)CH_2OH\) is named
as \textbf{2-hydroxomethyl-butanediol} and \tiny
\chemfig{-[:-30]-[:30](-[:-30]OH)-[:90](-[:30]COOH)-[:150]-[:-150]-[:-90]} \normalsize
is named as \textbf{2-hydroxo-1-cyclohexane carbonic acid}.

\section{Ethers}

Evaluate the two parts of the ether separated by the atom O, and pick the simplier one
(usually alkanes), and name it as \textbf{Simplier Part Prefix - oxy Complex Part}.
For example, compound \(CH_3CH_2OCH=CH_2\) can be expressed as \textbf{ethoxy ethene}.
and \(CH_3O-Ph\) \textbf{methoxy benzene}.

When one of the parts is cyclic hydrocarbons, and that the ring has less than 5 carbon
atoms, we can omit the 'yl' ending in the prefix.

\begin{table}[htbp]
	\centering
	\caption{Ether Prefix Word}
	\begin{tabular}{cl}
		\toprule 
		Chemical & Nomenclature \\
		\midrule 
        \(CH_3CH_2O-\)  & ethoxy \\
		\chemfig{*4(--(-O-)--)} & cyclobutoxy \\
		\chemfig{*6(--(-O-)----)} & cyclohex\textbf{yl}oxy \\
    \toprule
    \end{tabular}
\end{table}

Cyclic ethers are named firstly the alkane chain without the ether's oxygen atom,
and then specify the oxygen atom adding the \textbf{epoxy-} prefix.

\begin{table}[htbp]
	\centering
	\caption{Cyclic Ethers}
	\begin{tabular}{cl}
		\toprule 
		Chemical & Nomenclature \\
		\midrule 
		\chemfig{*3(-O-(-)-)} & 1,2-epoxypropane \\
		\chemfig{*4(-O---)} & 1,3-epoxypropane \\
		\chemfig{*5(-O----)} & 1,4-epoxybutane \textbf{tetrahydrofuran} \\
    \toprule
    \end{tabular}
\end{table}

Note: \textbf{Its index-marking rules are similar to alcohols.}

\section{Carboxylic Acids}

By replacing the last letter e in the corresonding alkanes, alkenes, and alkynes that
form the main chain, plus \textbf{-oic acid}. For acids that have more than one -COOHs
change the -oic ending into -dioic, -trioic etc.

Another nomenclature for carboxylic acids are \textbf{Name of Hydrocarbons + 
[(di, tri, etc)]carboxylic acid}. And many other organic acids have their common names.

\begin{table}[htbp]
	\centering
	\caption{Carboxylic Acids}
	\begin{tabular}{cll}
		\toprule 
		Chemical & Chain & Carboxylic Acids \\
		\midrule 
		\(CH_3CH=CHCH_2COOH\) & 3-pentene & 3-pentenoic acid \\
		\(CH_3CH_2CH(COOH)CH_2CH(COOH)CH_3\) & hexane & 2-ethyl-4-methyl-oentanedioic acid \\
		\chemfig{*6(-=(-COOH)-(-COOH)=-=)} & \normalsize benzene & 1,2-benzene dicarboxylic acid \\
    \toprule
    \end{tabular}
\end{table}

\begin{table}[htbp]
	\centering
	\caption{Common Names}
	\begin{tabular}{cl}
		\toprule 
		Chemical & Nomenclature \\
		\midrule 
		\(CH_3COOH\) & acetic acid \\
		\(HCOOH\) & formic acid \\
		\chemfig{*6(-=(-COOH)-=-=)} & benzoic acid \\
    \toprule
    \end{tabular}
\end{table}

Several generic types of organic compounds are seen as derivatives of carboxylic acids. And they
are named by replacing part of the acid ending into their own suffixes. They include 
\textbf{Nitriles}, \textbf{Amides}, \textbf{Halocarbonyls} and \textbf{Anhydrates}. Nitriles are
compounds that replace the -COOR group into -CN group, and is named by replacing \textbf{-ic acid}
into \textbf{-nitrile}. \emph{However, it seems that -anenitrile and -anonitrile are both acceptable.}
Amides are nitrigen based groups connected directly to the carbonyl, and are named by replacing
\textbf{-oic acid} into \textbf{-amide}. Halocarbonyls are halogens directly connected to carbonyl and
named by replacing \textbf{-ic acid} into \textbf{-yl halide}. Anhydrates are two carboxylic acid
eliminated by a water molecule, (\chemfig{-[:30](=[:90]O)(-[:-30]OH)} becames \chemfig{-[:30](=[:90]O)(-[:-30]O-[:30](=[:90]O)(-[:-30]))})
named by replacing \textbf{acid} to \textbf{anhydrate}. The example molecule is \textbf{Ethanoic Anhydrate, Acetic Anhydrate}.

Note: \textbf{Its index-marking rules are similar to alcohols.}

\section{Aldehydes and Ketones}

Replacing the last letter e of the names of hydrocarbons with the same amount of carbon
atoms with \textbf{-al} ending for aldehydes and \textbf{-one} for ketones.
For more complexed aldehydes, use the name for the corresponding carboxylic acids and 
replace the \textbf{-ic acid} ending with \textbf{-aldehyde}. 

\begin{table}[htbp]
	\centering
	\caption{Naming Aldehydes and Ketones}
	\begin{tabular}{clcl}
		\toprule 
		Hydrocarbons & Name & Aldehydes or Ketones & Name \\
		\midrule 
		\(CH_3CH(CH_3)CH3\) & 2-methylpropane & \(CH_3CH(CH_3)CHO\) & 2-methylpropanal \\
		\(CH_3(CH_2)_3CH_3\) & pentane & \(CH_3CO(CH_2)_2CH_3\) & 2-pantanone \\
    \toprule
    \end{tabular}
\end{table}

There is also a way to name ketones with the \(R_1 - CO - R_2\), for example, 
\textbf{propantone} can be called \textbf{dimethyl ketone} and \textbf{2-pantanone} 
\textbf{methyl propyl ketone}. For ketones directly connected to benzene, \chemfig{*6(-=(-COR)-=-=)}
you should find out the name of the carboxylic acid \(RCOOH\) and replace the \textbf{-oic acid}
ending with \textbf{-ophenone}

\section{Esters}

The alcoholic part of the ester is seen as a substituent and end with -yl hydrocarbonyl
ending, and the acid part replace the \textbf{-ic acid} ending with \textbf{-ate},
indicating the acid part of a salt. For example, \tiny \chemfig{*6(-=(-COOC_4H_9)-=-=)}
\normalsize is named as \textbf{Butyl benzoate}

\section{Amines}

For simple amines, consider using a hydrocarbonyl prefix and the -amine ending to name
the substance. System nomenclature (1) evaluate the longest carbon chain that contains
the specified nitrigen as main chain. (2) Mark the main carbon chain with smaller
indices closer to the nitrogen. (3) Naming the branched chain using the carbon index
and, if directly linked to the nitrogen, use capital letter N. (4) Replace the alkane
ending -e with -amine.

If there are more than one nitrogen in the compounds, -amine becomes -diamine, -triamine,
etc. For amides, see Carboxylic Acid section.

\begin{table}[htbp]
	\centering
	\caption{Sample Amine Names, \textbf{System Names}}
	\begin{tabular}{cl}
		\toprule 
		Chemical & Nomenclature \\
		\midrule 
		\((CH_3)_2NH\) & dimethylamine or \textbf{N-methylmethanamine} \\
		\((CH_3)_3N\) & trimethylamine or \textbf{N,N-dimethylmethanamine} \\
		\(CH_3(CH_2)_3-N(CH_3)_2\) & \textbf{N,N-dimethylbutanamine} \\
		\chemfig{*6(-=(-NH_2)-=-=)} & aniline or \textbf{banzenamine} \\
		\chemfig{*6(-=(-N(-[:30]CH_3)(-CH_2CH_3))-=-(-CH_3)=)} & \textbf{N-ethyl-N,4-dimethyl}benzenamine \\
    \toprule
    \end{tabular}
\end{table}

\section{Nitriles}

\(-CN\) as a maternal body is called \textbf{nitrile}s, while as a substituent is called
\textbf{cyano-, cyanide}. Firstly, regard the \(-CN\) as \(-COOH\) to name it as an 
carboxylic acid, then, replace the \textbf{-oic acid} ending with \textbf{-onitrile}.
For example, \textbf{Benzonic Acid} becomes \textbf{Benzonitrile}.

\section{Organometallic Compounds}

\begin{table}[htbp]
	\centering
	\caption{Place Organic Groups Before the Metal Atom}
	\begin{tabular}{cl}
		\toprule 
		Chemical & Nomenclature \\
		\midrule 
		\chemfig{CH_3Li} & Methylcopper \\
		\((C_2H_5)_2Hg\) & Diethylmercury \\
    \toprule
    \end{tabular}
\end{table}

\begin{table}[htbp]
	\centering
	\caption{Regard as Derivatives of Borane, Silane and Stannane}
	\begin{tabular}{cl}
		\toprule 
		Chemical & Nomenclature \\
		\midrule 
		\((CH_3)_4Si\) & Tetramethyl\textbf{silane} \\
		\((CH_3)_3SnC_2H_5\) & Ethyltrimethyl\textbf{stanname} \\
    \toprule
    \end{tabular}
\end{table}

For organic salt and positive metal ions, name them as a salt. For example, 
\(CH_3CH_2HgCl\) is named \textbf{ethylmercury chloride}

\section{Heterocyclic Compounds}

\subsection{System Names for Single Heterocycle}

(1) The suffix of the heterocyclic compounds is determined by the size of the cyclic 
ring, (2) and the non-carbon atoms that form the heterocycle has special prefixes
arranged in a certain order (O, S, N). (3) Modify the name ending with the following rules

\begin{itemize}
	\item If the ring contains \textbf{Nitrogen}, add an -e to the ending.
	\item If the ring doesn't contain \textbf{Nitrogen}, and is \textbf{saturated}, replace the last letter -e (if any) with -ane.
	\item If the ring contains \textbf{Nitrogen}, and is \textbf{saturated}, replace the last letter -e(or -ine for -irine, -epine, -ocine) (if any) with -idine.
	\item If the name contains to adjacent vowels, replace them to the last vowel. 
\end{itemize}

\begin{table}[htbp]
	\centering
	\caption{Cycle Size In Atoms and Corresponding Suffixes}
	\begin{tabular}{cl}
		\toprule 
		Size & Suffix \\
		\midrule 
		3 & -irine \\
		4 & -ete \\
		5 & -ole \\
		6 & -in(e) \\
		7 & -epin(e) \\
		8 & -ocin(e) \\
    \toprule
    \end{tabular}
\end{table}

\begin{table}[htbp]
	\centering
	\caption{Non-carbon Atom Prefix}
	\begin{tabular}{cl}
		\toprule 
		Atom & Prefix \\
		\midrule 
		O & oxa- \\
		S & thia- \\
		N & aza- \\
    \toprule
    \end{tabular}
\end{table}

\begin{table}[htbp]
	\centering
	\caption{Sample Names for Heterocyclic Compounds}
	\begin{tabular}{cl}
		\toprule 
		Structure & Naming \\
		\midrule 
		\chemfig{*4(-S-=-)} & \textbf{Naming Process:} \\
		& Thia-(1) + Ete \\
		& Thiete \\

		\chemfig{*4(-N=-=)} & Azete \\
		\chemfig{*5(-O--O-=)} & 1,3-di\textbf{ox}ole \\

		\chemfig{*3(-N--)} & \textbf{Naming Process:} \\
		& Aza-(1) + Ir(in) + (Nitrogen) (e) + (Saturated) Idine \\
		& Azairidine \\

		\textbf{Multiple Heteroatoms} & \\
		\chemfig{*5(-S-N-S-=)} & \textbf{Naming Process} \\
		& Thia-(1,3) + Aza-(2) + -Ole(5) \\
		& 1,3,2-Dithiazole \\
    \toprule
    \end{tabular}
\end{table}

\textbf{Hydrogen Indication} Mark out all the saturated carbon (with 2 hydrogen atom).
These marked indices are call indicating hydrogen. It is expressed using \textbf{Index
 + H} before the name of the compound.

\begin{table}[htbp]
	\centering
	\caption{Sample Names for Indicator Hydrogen}
	\begin{tabular}{cl}
		\toprule 
		Structure & Naming \\
		\midrule 
		\chemfig{*5(-N-=-=)} & 1H-Pyrrole \\
		\chemfig{*5(=N--=-)} & 2H-Pyrrole \\
		\chemfig{*5(-N=--=)} & 3H-Pyrrole \\
    \toprule
    \end{tabular}
\end{table}

\subsection{Common Names for Heterocycles}

\begin{table}[htbp]
	\centering
	\caption{Common Names}
	\begin{tabular}{cl}
		\toprule 
		Structure & Naming \\
		\midrule 
		\chemfig{*5(-O-=-=)} & Furan \\
		\chemfig{*5(-S-=-=)} & Thiophene \\
		\chemfig{*5(-N-=-=)} & Pyrrole \\
		\chemfig{*5(-S-=N-=)} & Thiazole \\
		\chemfig{*5(-N(-H)-=N-=)} & Imidazole \\
		\chemfig{*6(=N-=-=-)} & Pyridine \\
		\chemfig{*6(-O-=--=)} & Pyran \\
		\chemfig{*6(=N-=N-=-)} & Pyrimidine \\
    \toprule
    \end{tabular}
\end{table}

\setchemfig {
	bond sep = 1.1pt,
	atom sep = 2em,
	bond style = {line width = 0.5pt}
}

\subsection{Common Names for Condensed Heterocyclic Compounds}

Nearly all of the condensed heterocyclic compounds have a common names, and the
table demonstrates some of the most commonly used compounds.

\begin{table}[htbp]
	\centering
	\caption{Common Names for Condensed Heterocyclic Compounds}
	\begin{tabular}{cl}
		\toprule 
		Molecule & Name \\
		\midrule 
		\chemfig{*6(=-(*5(-O-=--))=-=-)} & Benzofuran \\
		\chemfig{*6(=-(*5(-S-=--))=-=-)} & Benzothiophene \\
		\chemfig{*6(=-(*5(-N(-H)-=--))=-=-)} & Indole \\
		\chemfig{*6(=-(*6(-N=-=--))=-=-)} & Quinoline \\
		\chemfig{*6(=-(*6(-=N-=--))=-=-)} & Isoquinoline \\
    \toprule
    \end{tabular}
\end{table}

\subsection{Condensed Heterocyclic Compounds Nomenclatures}

\textbf{Step 1: Marking the Edge and Atoms in the Heterocycle} Every condense heterocycle
can be splitted into many small unit cycles. The first step of expressing the whole is
to express all of the subunits to mark the relative position in which they stack. The 
marking should follow these rules:

\textbf{Rules to Determine Indices}

\begin{itemize}
	\item Make the indices of all hetero-atoms the \textbf{smallest}.
	\item The carbon with no hydrogen should't be assigned with an index, assign them the previous instead with mark a,b,etc. instead. For example, 1-2-3-3a-3b-4.
	\item Among all the circumstances that match the first rule, try to make those carbons with no hydrogen's indices the \textbf{smallest}.
	\item Among all the circumstances that match the rule 1 and 3, try to make the indices of all the indication hydrogen the \textbf{smallest}.
	\item Priority: O, S, NH, N
	\item Assign the index (a, b, c, etc) of the line contours of the heterocycle according to the indices of atoms.  
\end{itemize}

\begin{table}[htbp]
	\centering
	\caption{Step 1: Rule 1}
	\begin{tabular}{cc}
		\toprule 
		Correct & Wrong \\
		\midrule 
		\chemfig{*4(_4-N_1-_2-_3-)} & \chemfig{*4(_1-N_4-_3-_2-)} \\
		Nitrogen has the smallest index 1 & \\
    \toprule
    \end{tabular}
\end{table}

\begin{table}[htbp]
	\centering
	\caption{Step 1: Rule 2 and Rule 3}
	\begin{tabular}{cc}
		\toprule 
		Correct & Wrong \\
		\midrule 
		\chemfig{*6(N_4=3-2=N_1-N_8(*5(N_8-7=6-N_5=4a-))-4a-)} &
		\chemfig{*6(N_1=2-3=N_4-N_5(*5(N_5-6=7-N_8=8a-))-8a-)} or 
		\chemfig{*6(N_8=7-6=N_5-N_4(*5(N_4-3=2-N_1=8a-))-8a-)} \\
		4a is smaller than 8a & \\
    \toprule
    \end{tabular}
\end{table}

\textbf{Step 2: Determine the Fundamental Ring} Among all the subunits of the heterocycle,
select the one which matched the following rules as the fundamental ring.

\textbf{Rules to Determine Fundamental Ring}

\begin{itemize}
	\item If the subunit rings contains both aromatic rings and heterotic rings, consider the \textbf{heterotic ring} as the fundamental ring. If there are more than one heterotic rings, choose the \textbf{larger one} or \textbf{with a special name}.
	\item If the subunit rings contains only heterotic rings, choose the \textbf{larger ring}.
	\item If the size of the rings are identical, choose the one with \textbf{most heterotic atom}.
	\item If the number of heterotic atoms are identical, choose the one with \textbf{maximum types of atoms}.
	\item If all of these are identical, choose the one with \textbf{smallest indices} of heterotic atoms.
	\item In the naming of each subunits, treat the shared heterotic atom as their own.
\end{itemize}

\textbf{Step 3: Naming} Now that we acquire all the subunit rings and their right sequences
of indices and contours, we can name it using relative positioning, in a bracket, first
sign the arrangement of the shared contour, than the line specified by a small letter.
[3,2-b] is a legal notation. However, [3,2-b] and [2,3-b] is not the same. [3,2] specifies
a contour of additional ring, and b specifies a contour of fundamental ring.

\begin{table}[htbp]
	\centering
	\caption{The Naming Process of a Condensed Ring}
	\begin{tabular}{cl}
		\toprule 
		Target Molecule &  \chemfig{*5(-S-(*5(=N-=-N-))-N-=)} \\
		\midrule 
		& \textbf{Naming Each Subunits} \\
		\chemfig{*5(-S-=N-=)} & Thiazole (Fundamental Ring) \\
		\chemfig{*5(=N-=-N(-H)-)} & 1H-1,3-Diazole, Imidazole (Additional Ring) \\

		& \textbf{Marking the Index Of Subunits} \\
		\chemfig{*5(5-S_1-2=N_3-4=)} & \\
		\chemfig{*5(2=N_3-4=5-N_1(-H)-)} & \\

		& \textbf{Naming the Compound} \\
		\chemfig{*5(-S-(*5(=N-=-N-))-N-=)} & Imidazo[2,1-b]thiazole \\
    \toprule
    \end{tabular}
\end{table}

\setchemfig {
	bond sep = 1.1pt,
	atom sep = 1.1em,
	bond style = {line width = 0.5pt}
}

\section{Common Functional Groups}

\begin{table}[htbp]
	\centering
	\caption{List of Functional Groups}
	\begin{tabular}{ccc}
		\toprule 
		Chemical Structure & Prefixes & Suffixes \\
		\midrule 
        \(-COOH\) & carboxy & -carboxylic acid, oic acid \\
        \(-SO_3H\) & sulfo & -sulfolic acid \\
		\(-COOR\) & R-oxycarbonyl & R...carboxylate, R...oate \\
		\(-COX\) & halocarbonyl & -carbonyl halide, -oyl halide \\
		\(-CONH_2\) & carbamoyl & -carboxamide, -amide \\
		\(-CN\) & cyano & -carbonitrile, -nitrile \\
		\(-CHO\) & formyl & -carbaldehyde, -al \\
		\(-CO\) & oxo & -one \\
		\(-OH\) & hydroxy & -ol \\
		        & & (phen)-ol \\
	    \(-NH_2\) & amino & -amine \\
		\(-OR\) & R-oxy & -ether \\
		\(-R\) & alkyl & \\
		\(-X\) & halo & \\
		\(-NO_2\) & nitro & \\
		\(-NO\) & nitriso & \\
    \toprule
    \end{tabular}
\end{table}

\textbf{The precedence of organic functional groups} should be considered when there
are multiple different groups in a compound. the topmost functional group should be 
regarded as the maternal body while the less precedent groups be substituents.

\textbf{Cahn-Ingold-Prelog Sequence Rule} Here we introduce a simplified version of
the sequence rule to arrange the relative size of different functional groups.

\textbf{Rule I} The sequence of common atoms used in organic compounds is listed in
descending order of size. \textbf{I, Br, Cl, S, P, F, O, N, C, D, H, $\emptyset$}.

\textbf{Rule II} Given that a functional group has only one atom that are directly
connected. We start from that atom, and express all the atoms that are directly 
connected to it in descending size order. This forms a triary tree structure. For
example, \(-CH_3\) is expressed as C(H,H,H) and \(-CHFCl\) is expressed as C(Cl,F,H).
If the atom has double bond or triple bond to connect with an atom, if is regarded 
as two or three identical bond to the atom. For example, \(-CHO\) is expressed as C(O,O,H).
If the saturate bond is less than four(for non-carbon center atoms), we fill the
tree with empty imagery atoms. \(-OH\) is expressed as O(H,$\emptyset$,$\emptyset$).

\textbf{Rule III} If a single atom cannot get the sequential result of the groups,
you should extnds to the atom that follows. however, the imagery atoms, and those
atoms considered by equivalating double or triple bonds should be regarded as naked
atoms without further branch (to avoid iteration hell). \(-CN\) is regarded as 
C(N([C],[C],$\emptyset$),[N],[N]), \chemfig{*6(-=-(-)=-=)} is regarded as C(C(H,C,[C]),[C],C(H,C,[C]))

\begin{table}[htbp]
	\centering
	\caption{Precedence of Common Functional Groups}
	\begin{tabular}{cl}
		\toprule 
		Chemical & Nomenclature \\
		\midrule 
		\textbf{Higher Precedence} & \\
		\chemfig{-COOH} & Carboxylic Acids \\
		\chemfig{-SO_3H} & Sulfuric Acids \\
		\chemfig{-COOR} & Esters \\
		\chemfig{-COX} & Carbonyl Halides \\
		\chemfig{-CONH_2} & Carboxamides \\
		\chemfig{-CN} & Nitriles \\
		\chemfig{-CHO} & Aldehydes \\
		\chemfig{O=C(-[:60])(-[:-60])} & Ketones \\
		\chemfig{-OH} & Alcohols and Phenols \\
		\chemfig{-NH_2} & Amines \\
		\chemfig{*6(-=-=-=)} & Aromatic Rings \\
		\chemfig{-C~C-} & Alkynes \\
		\chemfig{-C=C-} & Alkenes \\ 
		\chemfig{-O-} & Ethers \\
		\chemfig{R} & Alkanes \\
		\chemfig{-X} & Halogens \\
		\chemfig{-NO_2} & Nitro \\
		\textbf{Lower Precedence} & \\
    \toprule
    \end{tabular}
\end{table}

\end{document}