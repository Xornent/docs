% !TEX TS-program = xelatex

\documentclass[10pt]{ctexart}

\usepackage{geometry}
\geometry{a4paper}
\usepackage{graphicx}

\title { \textbf{基因工程制药} }
\author {}
\begin{document}
\maketitle
% \setlength{\parskip}{0.8 \baselineskip}

\section{ 生物药物分类 }

\textbf{从使用用途分类}

\begin{itemize}
    \item 治疗:干扰素,胰岛素,生长激素,集落刺激因子
    \item 预防:疫苗
    \item 诊断:免疫诊断试剂,酶诊断试剂,单抗诊断试剂,基因诊断试剂
\end{itemize}

\textbf{按原料分类}

\begin{itemize}
    \item 人体组织来源:血液制品,白蛋白,细胞因子,血小板
    \item 动物来源:消化酶,动物激素,成骨蛋白,牛黄,蛇毒等
    \item 微生物来源:抗生素,核算,维生素,酶
    \item 植物来源:醌,黄酮,鞣质,甾体,萜类
    \item 海洋生物来源:甲壳素,鱼肝油,卵磷脂
\end{itemize}

\textbf{按生物化学性质}

\begin{itemize}
    \item 氨基酸类
    \item 醇和酮
    \item 维生素
    \item 酶和辅酶(消化酶类;消炎酶类;溶栓酶类:纤维溶解酶原激活剂;抗肿瘤酶;超氧化物)酶类的改性
    \item 脂肪酸:磷脂类,多不饱和脂肪酸
    \item 多肽和蛋白质类
    \item 核酸类及其衍生物(RNA-iRNA 干扰),化疗药物(核酸类似物:干扰癌细胞 DNA 复制)
    \item 多糖:硫酸软骨素,肝素,透明质酸,壳聚糖,灵芝多糖 \dots
\end{itemize}

传统生物药物由于来源和制备上的困难,且可能受到病毒,衣原体和支原体的感染。
使用安全实用可靠的制备方法

\begin{itemize}
    \item 可以大量生产生理活性蛋白
    \item 可以对蛋白质进行改性
    \item 可以扩增产物,便于研究
\end{itemize}

\textbf{基因治疗}

\begin{itemize}
    \item 如何选择合适的治疗基因
    \item 安全的(没有免疫排斥和抗体)载体
    \item 如何定向导入到靶细胞中
    \item 生物伦理问题
\end{itemize}

\end{document}